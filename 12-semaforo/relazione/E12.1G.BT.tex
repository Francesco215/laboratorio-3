\documentclass{article}
\usepackage[italian]{babel}
\usepackage[T1]{fontenc}
\usepackage{graphicx}
\usepackage[utf8x]{inputenc}
\usepackage{amsmath}
\usepackage{amsthm}
\usepackage{hyperref}
\usepackage{caption}
%\usepackage{tikz}
\date{}
\author{Gruppo 1G.BT \\Lorenzo Cavuoti, Francesco Sacco}
\title{Caratteristiche porte logiche e semplici circuiti logici}

\begin{document}
\maketitle
\paragraph{3)}
	\subparagraph{a)}
		mettere immagine carina
	\subparagraph{b)}
		Indicheremo uno stato $S=VG$ dove $V$ è lo stato del led verde, mentre $G$ è lo stato del led giallo. In totale gli stati sono 3: $10, 11, 00$. A prima vista può sembrare che non si itene conto dello stato del led rosso, tuttavia esso è univocamente determinato dallo stato verde in quanto ha sempre uno stato opposto al led verde.
	\subparagraph{c)}
		La tabella di verità è la seguente\newline
		\begin{tabular}{cc|cc}
			\hline
			$V_n & G_n & V_{n+1} & G_{n+1}$\\
			\hline
			$1 & 0 & 1 & 1$\\
			$1 & 1 & 0 & 0$\\
			$0 & 0 & 1 & 0$\\
			\hline
		\end{tabular}
	\subparagraph{d)}
				
\end{document}

		\begin{tabular}{cc|cc}
			\hline
			$V_n & G_n & V_{n+1} & G_{n+1}$\\
			\hline
			$1 & 0 & 1 & 1$\\
			$1 & 1 & 0 & 0$\\
			$0 & 0 & 1 & 0$\\
			\hline
		\end{tabular}