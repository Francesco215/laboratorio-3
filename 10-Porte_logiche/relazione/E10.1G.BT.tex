\documentclass{article}
\usepackage[italian]{babel}
\usepackage[T1]{fontenc}
\usepackage{graphicx}
\usepackage[utf8x]{inputenc}
\usepackage{amsmath}
\usepackage{amsthm}
\usepackage{hyperref}
\usepackage{caption}
\date{}
\author{Francesco Sacco Lorenzo Cavuoti}
\title{Caratteristiche porte logiche e semplici circuiti logici}

\begin{document}
\maketitle
\paragraph{0)}
	Lo scopo dell'esperienza è misurare le caratteristiche statiche e dinamiche delle porte NOT contenute nell’integrato SN74LS04 (HEX Inverter) e costruire semplici circuiti logici con le porte NAND.
	
\paragraph{1)}
	Si è montato il circuito in figura \ref{fig:circuito-1} e si è alimentato con $V_{CC}=4.7\pm0.2$ V usando solo un generatore. Successivamente si è fatta variare la resistenza del potenziometro e si è segnato $V_{in}$ e $V_{out}$ per ciascuna posizione del potenziometro, i dati sono riportati in tabella \ref{tab:124} e nel grafico in figura, le tensioni sono state misurate con i cursori dell'oscilloscopio.\newline

	\begin{minipage}{.6\linewidth}
		\centering
		\includegraphics[width=\linewidth]{figure/circuito1}
		\captionof{figure}{Circuito usato nel punto 1}
		\label{fig:circuito-1}
	\end{minipage}
	\begin{minipage}{.4\linewidth}
		\begin{tabular}{cc}
\hline
	$V_{in}[V]$ & $V_{out}[V]$\\ 
\hline
	$4.6\pm0.2$ & $0.134\pm0.006$ \\
	$4.2\pm0.2$ & $0.134\pm0.006$ \\
	$3.2\pm0.1$ & $0.134\pm0.006$ \\
	$2.5\pm0.1$ & $0.134\pm0.006$ \\
	$1.84\pm0.09$ & $0.134\pm0.006$ \\
	$1.26\pm0.05$ & $0.134\pm0.006$ \\
	$1.16\pm0.05$ & $0.134\pm0.006$ \\
	$1.06\pm0.05$ & $2.5\pm0.1$ \\
	$1.04\pm0.05$ & $2.0\pm0.09$ \\
	$0.74\pm0.03$ & $4.1\pm0.2$ \\
	$0.34\pm0.01$ & $4.2\pm0.2$ \\
	$0.144\pm0.006$ & $4.2\pm0.2$ \\
\hline
\end{tabular}

		\label{tab:124}
	\end{minipage}\newline\newline

	Usando il potenziometro è stato possibile stimare i voltaggi VOH,VOL,VIH,VIL che si possono vedere nella tabella qui sotto.\newline
	Di conseguenza le bande d'incertezza misurate d'input è $0.418\pm0.007$, mentre quella di datasheet è $1.2 V$; la barra d'incertezza misurata d'output è $3.95\pm0.02$ e quella di datasheet è $3.2 V$.
	\begin{center}
		\begin{tabular}{ccc}
\hline
	$V_S[V]$ & $V_{out}[V]$ & $V_{out}/V_S$\\ 
\hline
	$38\pm2\ mV$ & $125\pm5\ mV$ & $3.4\pm0.2$ \\
	$0.101\pm0.005$ & $0.3\pm0.01$ & $3.0\pm0.2$ \\
	$0.23\pm0.01$ & $0.68\pm0.03$ & $2.7\pm0.2$ \\
	$0.51\pm0.02$ & $1.24\pm0.05$ & $2.6\pm0.2$ \\
	$0.75\pm0.03$ & $1.93\pm0.09$ & $2.8\pm0.2$ \\
\hline
\end{tabular}

	\end{center}

\paragraph{2)}
	Per il secondo punto abbiamo montato all'uscita del NOT una resistenza di $3.31\pm0.03 k\Omega$ verso $V_{CC}$ e abbiamo mandato all'ingresso del NOT un'onda quadra di ampiezza tra $0$ e $5.0\pm0.2 V$ con una frequenza di circa 1kHz.
	I tempi misurati sono $t_{PHL}\approx7.2ns$ (quello di datasheet è $10ns$), mentre $t_{PLH}\approx55ns$ (quello di datasheet è $25ns$).
	I dati presentano un'elevata incertezza perchè il segnale era parecchio rumoroso, come si più vedere dalle immagini \ref{fig:THL} e \ref{fig:TLH}.
	\begin{minipage}{.5\linewidth}
		\centering
		\includegraphics[width=\linewidth]{figure/THL}
		\captionof{figure}{$t_{PHL}$}
		\label{fig:THL}
	\end{minipage}
	\begin{minipage}{.5\linewidth}
		\centering
		\includegraphics[width=\linewidth]{figure/TLH}
		\captionof{figure}{$t_{PLH}$}
		\label{fig:TLH}
	\end{minipage}
	
\paragraph{3)}
	\subparagraph{a.}
	Si è montato il circuito in figura \ref{fig:NAND} dove i segnali A, B sono dati da uno DIP Switch a 4 interuttori con l'altro estremo collegato a massa.
	\subparagraph{b.}
	Per verificare velocemente la tabella di verità si è collegato un led all'uscita dei circuiti logici, cosi' da notare il passaggio di corrente senza dover misurare con il multimetro o l'oscilloscopio.
	\begin{figure}
	\begin{center}
		\includegraphics[width=0.5\linewidth]{figure/NAND}
		\captionof{figure}{Schema circuitale NAND}
		\label{fig:NAND}
	\end{center}
	\end{figure}
	\subparagraph{c.}
	\begin{description}
		\item[i.] Per costruire la porta AND (figura \ref{fig:AND}) si sono utilizzate 2 porte NAND. Prima si esegue A NAND B, successivamente si nega il segnale in uscita cosi' da ottenere A AND B
		\item[ii.] Per la porta OR (figura \ref{fig:OR}) si sono utilizzate 3 porte NAND, infatti sfruttando le leggi di De Morgan si ha: A OR B = (!A) NAND (!B) 
		\item[iii.] Per la porta XOR (figura \ref{fig:XOR}) si sono utilizzate 4 porte NAND,  il circuito è sempre stato ricavato dalle leggi di De Morgan
		\item[iv.] Per il sommatore ad un bit (figura \ref{fig:sommatore}) si sono utilizzate 5 porte NAND, anche se il circuito risulta abbastanza complicato in realtà questo è composto  da uno XOR, che da la seconda cifra, e da un AND che da la prima cifra della somma.\newline
	\end{description}

	\begin{minipage}{.5\linewidth}
		\includegraphics[width=\linewidth]{figure/AND1}
		\captionof{figure}{Schema circuitale AND}
		\label{fig:AND}
	\end{minipage}
	\begin{minipage}{.5\linewidth}
		\centering
		\includegraphics[width=\linewidth]{figure/OR1}
		\captionof{figure}{Schema circuitale OR}
		\label{fig:OR}
	\end{minipage}\newline
	\begin{minipage}{.5\linewidth}
		\includegraphics[width=\linewidth]{figure/XOR1}
		\captionof{figure}{Schema circuitale XOR}
		\label{fig:XOR}
	\end{minipage}
	\begin{minipage}{.5\linewidth}
		\includegraphics[width=\linewidth]{figure/SOMMA1}
		\captionof{figure}{Schema circuitale sommatore}
		\label{fig:sommatore}
	\end{minipage}\newline

\end{document}