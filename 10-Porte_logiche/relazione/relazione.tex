\documentclass{article}
\usepackage[italian]{babel}
\usepackage[T1]{fontenc}
\usepackage{graphicx}
\usepackage[utf8x]{inputenc}
\usepackage{amsmath}
\usepackage{amsthm}
\usepackage{hyperref}
\usepackage{caption}
\date{}
\author{Francesco Sacco Lorenzo Cavuoti}
\title{Caratteristiche porte logiche e semplici circuiti logici}

\begin{document}
\maketitle
\paragraph{0)}
	Lo scopo dell'esperienza è misurare le caratteristiche statiche e dinamiche delle porte NOT contenute nell’integrato SN74LS04 (HEX Inverter) e costruire semplici circuiti logici con le porte NAND.
	
\paragraph{1)}
	Si è montato il circuito in figura \ref{fig:circuito-1} e si è alimentato con $V_{CC}=4.7\pm0.2$ V usando solo un generatore. Successivamente si è fatta variare la resistenza del potenziometro e si è segnato $V_{in}$ e $V_{out}$ per ciascuna posizione del potenziometro, i dati sono riportati in tabella e nel grafico in figura.\newline

	\begin{minipage}{.6\linewidth}
			\centering
			\includegraphics[width=\linewidth]{figure/circuito1}
			\captionof{figure}{Circuito 1}
			\label{fig:circuito-1}
	\end{minipage}
	\begin{minipage}{.4\linewidth}
		\begin{tabular}{cc}
\hline
	$V_{in}[V]$ & $V_{out}[V]$\\ 
\hline
	$4.6\pm0.2$ & $0.134\pm0.006$ \\
	$4.2\pm0.2$ & $0.134\pm0.006$ \\
	$3.2\pm0.1$ & $0.134\pm0.006$ \\
	$2.5\pm0.1$ & $0.134\pm0.006$ \\
	$1.84\pm0.09$ & $0.134\pm0.006$ \\
	$1.26\pm0.05$ & $0.134\pm0.006$ \\
	$1.16\pm0.05$ & $0.134\pm0.006$ \\
	$1.06\pm0.05$ & $2.5\pm0.1$ \\
	$1.04\pm0.05$ & $2.0\pm0.09$ \\
	$0.74\pm0.03$ & $4.1\pm0.2$ \\
	$0.34\pm0.01$ & $4.2\pm0.2$ \\
	$0.144\pm0.006$ & $4.2\pm0.2$ \\
\hline
\end{tabular}

		\label{tab:124}
	\end{minipage}\newline\newline

	Usando il potenziometro è stato possibile stimare i voltaggi VOH,VOL,VIH,VIL che si possono vedere nella tabella qui sotto.\newline
	Di conseguenza le bande d'incertezza misurate d'imput è $0.418\pm0.007$, mentre quella di datasheet è $1.2 V$; la barra d'incertezza misurata d'output è $3.95\pm0.02$ e quella di datasheet è $3.2 V$.
	\begin{center}
		\begin{tabular}{ccc}
\hline
	$V_S[V]$ & $V_{out}[V]$ & $V_{out}/V_S$\\ 
\hline
	$38\pm2\ mV$ & $125\pm5\ mV$ & $3.4\pm0.2$ \\
	$0.101\pm0.005$ & $0.3\pm0.01$ & $3.0\pm0.2$ \\
	$0.23\pm0.01$ & $0.68\pm0.03$ & $2.7\pm0.2$ \\
	$0.51\pm0.02$ & $1.24\pm0.05$ & $2.6\pm0.2$ \\
	$0.75\pm0.03$ & $1.93\pm0.09$ & $2.8\pm0.2$ \\
\hline
\end{tabular}

	\end{center}

\paragraph{2)}
	Per il secondo punto abbiamo usato una resistenza di $3.31\pm0.03 k\Omega$, una frequenza di circa $1kHz$ e abbiamo mandato un'onda quadra di $5.0\pm0.2 V$.
	i tempi misurati sono $t_{PHL}=(7.2\pm0.8)ns$ (quello di datasheet è $10ns$), mentre $t_{PLH}=55\pm 1 ns$ (quello di datasheet è $25ns$).
	
	

	




\end{document}