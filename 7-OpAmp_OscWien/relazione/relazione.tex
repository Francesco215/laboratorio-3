\documentclass{article}
%\documentclass{exam}
\usepackage[italian]{babel}
\usepackage[T1]{fontenc}
\usepackage{graphicx}
\usepackage[utf8x]{inputenc}
\usepackage{amsmath}
\usepackage{amsthm}
\usepackage{hyperref}
\date{Dicembre 2018}
\author{Francesco Sacco}
\title{Oscillatore di Wien}

\begin{document}
\maketitle
\paragraph{1)}
	Per il primo punto ho usato un segnale in ingresso $V_s$ con un ampiezza picco picco di $260\pm11 mV$, e ho fatto delle misurazioni con dei segnali con frequenza compresa tra i $500$Hz e $3$kHz. I valori delle misure e i grafici sono riportati qui sotto\newline
	\begin{center}
		\begin{tabular}{cccc}
\hline
	$f$[kHz] & $V_A$[mV] & $V_A/V_{in}[dB]$ & fase [gradi]\\ 
\hline
	$0.4495\pm0.0001$ & $(1.25\pm0.05)\times 10^{2}$ & $-6.4\pm0.5$ & $43.4\pm0.9$ \\
	$0.6811\pm0.0001$ & $(1.55\pm0.07)\times 10^{2}$ & $-4.5\pm0.5$ & $29.4\pm0.6$ \\
	$1.00467\pm0.00001$ & $(1.72\pm0.09)\times 10^{2}$ & $-3.6\pm0.6$ & $15.9\pm0.3$ \\
	$1.2169\pm0.0001$ & $(1.78\pm0.09)\times 10^{2}$ & $-3.3\pm0.6$ & $9.0\pm0.2$ \\
	$1.59834\pm0.00001$ & $(1.82\pm0.09)\times 10^{2}$ & $-3.1\pm0.6$ & $0\pm 8.0\times 10^{-2}$ \\
	$2.17434\pm0.00001$ & $(1.74\pm0.09)\times 10^{2}$ & $-3.5\pm0.6$ & $-11.6\pm0.2$ \\
	$2.89413\pm0.00001$ & $(1.66\pm0.08)\times 10^{2}$ & $-3.9\pm0.6$ & $-24.2\pm0.5$ \\
\hline
\end{tabular}
\\
	\end{center}
	\begin{figure}[h]
		\centering
		\includegraphics[width=\linewidth]{figure/1.png}
		\caption{sfasamento del segnale in funzione della frequenza in ingresso}
		\label{fig:1}
	\end{figure}
	\begin{figure}[h]
		\centering
		\includegraphics[width=\linewidth]{figure/1res.png}
		\caption{Residui dello sfasamento}
		\label{fig:1res}
	\end{figure}
	\begin{figure}
		\centering
		\includegraphics[width=\linewidth]{figure/2.png}
		\caption{Attenuazione del segnale in funzione della frequenza in infresso}
		\label{fig:2}
	\end{figure}
	dalla figura \ref{fig:1} si evince chiaramente che lo sfasamento diminuisce al'aumentare della frequenza e si ha uno zero alla frequenza di taglio $f_t=1/2\pi\sqrt{R_1R_2C_1C_2}$, i valori delle componenti indicate nel circuito in figura \ref{fig:circ1} sono disponibili nella lista qui sotto
	\begin{itemize}
		\item $R_1=9.99\pm0.08k\Omega$
		\item $R_2=9.91\pm0.08k\Omega$
		\item $C_1=11.1\pm0.4nF$
		\item $C_2=9.8\pm 0.4nF$
	\end{itemize}
	Ne consegue che $f_t=1.5\pm0.4 k$Hz\footnote{L'errore risulta grande perchè quando ho propagato $R_1R_2C_1C_2$ sull'inverso della radice, essendo la derivata intorno allo zero molto alta l'errore è esploso. Infatti\newline $R_1R_2C_1C_2=(1.07\pm 0.06)\times 10^{-8}$}. Essendo in un regime con frequenze vicine a quella di taglio mi sono permesso di fare un fit lineare per vedere nel grafico dei residui gli errori (fig \ref{fig:1res})\newline
	\begin{figure}
		\centering
		\includegraphics[width=\linewidth]{figure/circ1.png}
		\caption{Circuito 1}
		\label{fig:circ1}
	\end{figure}
	Per quanto riguarda l'attenuazione si nota dall'immagine \ref{fig:2} un massimo nella frequenza di taglio, mentre le altre frequenze vengono attenuate di più il valore teorico dell'attenuazione è $A\times \beta(f)$ dove $A$ è l'attenuazione del partitore di tenzione messo a feedback, mentre $\beta(f)$ è l'attenuazione del partitore di tenzione generalizzato composto dalle resistenze e capacità.\newline
\paragraph{2)}
	girando il potenziometro il segnale aumenta o diminuisce di ampiezza, finchè a un certo punto scompare, perchè $|A\beta|<1$
\paragraph{3)}
	La frequenza varia in modo insignificante al variare della posizione del potenziometro, come detto prima l'ampiezza è il parametro che varia di più al variare dell'oscillazione
\paragraph{4)}
	Misurando con l'oscilloscopio i voltaggi si ottiene che $V_{A}=242\pm1mV$ e $V_{out}=716\pm3mV$, facendo il rapporto si ottiene $V_{out}/V_{A}=2.9\pm0.2$, che è in linea con la teoria
\paragraph{5)}
	Succede che non si trovano più nel circuito
\end{document}